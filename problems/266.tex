
\taskpic[3cm]{Пластмассовый кубик со стороной 10~см привязан к невесомой
  нерастяжимой нити, которая намотана на катушку. Разматывая катушку,
  кубик погружают в бассейн с жидкостью. Плотность жидкости зависит от
  глубины. График этой зависимости представлен на рисунке. В самом
  начале погружения нижняя грань кубика касается жидкости. Постройте
  график зависимости силы натяжения нити от длины ее размотанной
  части. Плотность пластмассы, из которой сделан куб, равна $\rho =
  1350$~кг/м$^3$.}
{
  \begin{tikzpicture}
    \draw[thin,fill=blue!20] (0,0) rectangle (3,1); 
    \draw[very thick] (0,2) -- (0,0) -- (3,0) -- (3,2);
    \draw[thick] (1,2.5) circle (0.5cm);
    \draw[very thick] (1,2.5) -- (1,3.3);
    \draw[thick,interface] (0.5,3.3) -- (1.5,3.3);
    \draw[thick] (1.5,2.5) -- (1.5,1.1);
    \draw[fill=white] (1.25,1.1) rectangle (1.75,0.6);
  \end{tikzpicture}
}

\begin{center}
  \begin{tikzpicture}
    \begin{axis}[grid=both, xmin=0, ymin=1150, ymax=1450,
      y=0.015cm, x=0.25cm, xlabel={Глубина, см},ylabel={Плотность,
        $\mbox{кг/м}^3$},ytick={1200,1250,...,1450}]
      \addplot[mark=none,very thick,blue,line width=2pt,domain=0:45] {1200+5*x};
    \end{axis}
  \end{tikzpicture}
\end{center}
% Город-2004, 7 класс
