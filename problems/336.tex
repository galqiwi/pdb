\task{ Два велосипедиста одновременно выезжают навстречу друг другу из
  деревень Егорово и Серово, находящихся на расстоянии $L=10$ км друг
  от друга. Каждый планирует ехать со скоростью $V=20$ км/ч и,
  достигнув противоположной деревни, сразу повернуть обратно. Но вдоль
  дороги всё время дует ветер, скорость и направление которого
  постоянны. При движении по ветру скорость увеличивается на столько
  же, на сколько уменьшается при движении против ветра. Велосипедист,
  который сначала ехал по ветру, достигнув противоположной деревни,
  сразу повернул назад, а велосипедист, который сначала ехал против
  ветра, задержался в противоположной деревне, чтобы отдохнуть, и
  только потом поехал обратно. Известно, что велосипедисты встречались
  в точках \textbf{A} и \textbf{B}, находящихся на расстоянии $L_A =
  2$ км и $L_B=6$ км от Егорово. Найдите времена движения обоих
  велосипедистов. Найдите также время, которое потратил на отдых
  уставший велосипедист. }
% Москва-2007, 8 класс
% ccpe-2016-2017-8