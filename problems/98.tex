\taskpic{
	За линзой на расстоянии $\ell = 4 \unit{см}$ (больше фокусного) расположено перпендикулярно главной
    оптической оси плоское зеркало. Перед линзой, также перпендикулярно главной оптической оси,
    расположен лист клетчатой бумаги. На это листе получают изображение его клеток при двух положениях
    листа относительно линзы. Эти положения отличаются на $L = 9 \unit{см}$. Определить фокусное
    расстояние линзы.
}{
	\begin{tikzpicture}
    	\draw[very thick] (0, 0) -- (3.5, 0);
        \draw[very thick] (1, 1.1) -- (1, -1.1);
        \draw[very thick, <->] (2.5, 1.1) -- (2.5, -1.1);
        \draw[very thick, interface] (3.5, 1.1) -- (3.5, -1.1);
        \draw[blue, <->] (2.5, -1.2) -- (3.5, -1.2) node[midway, below] {$\ell$};
    \end{tikzpicture}
}