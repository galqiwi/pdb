\taskpic{ В герметичном сосуде сверху находится жидкость с плотностью
  $\rho_0 = 800 \kgm$, отделённая лёгким подвижным поршнем от газа,
  находящегося внизу и имеющего давление $p=20\unit{кПа}$. В поршне
  есть круглое отверстие, в которое вставлен цилиндрический
  поплавок. В жидкость поплавок погружен на длину $h$, а в газ на
  длину $3h$. Площадь основания поплавка $S$. Поплавок может свободно
  скользить относительно поршня, а поршень относительно стенок сосуда.
  Жидкость нигде не подтекает. Какой должна быть плотность поплавка
  $\rho$, чтобы система могла оставаться в равновесии? Ускорение
  свободного падения $g = 10\mc$. }
{
  \begin{tikzpicture}
    \draw[thick] (0,0) rectangle ++(4,5);
    \draw[fill=gray!70] (0,3.3) rectangle ++(1.5,0.1);
    \draw[fill=gray!70] (4,3.3) rectangle ++(-1.5,0.1);
    \draw[fill=gray!20] (1.5,4) rectangle ++(1,-2.8);
    \draw[blue,thick,<->] (1.7,4) -- ++(0,-0.7) node[midway,right]
    {$h$}; 
    \draw[blue,thick,<->] (1.7,3.3) -- ++(0,-2.1) node[midway,right=-0.1cm]
    {$3h$};
    \draw (0.2,0) node[anchor = south west] {\normalsize газ};
    \draw (0.2,4.4) node[anchor = south west] {\normalsize жидкость};
    \draw (3.7,4.7) node {\normalsize $\rho_0$};
    \draw (3.7,0.3) node {\normalsize $p$}; 
  \end{tikzpicture}
}
% Максвелл-2014, 8 класс
% ccpe-2016-2017-8