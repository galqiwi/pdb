\task{
	Легкая нерастяжимая нить длиной $2 \unit{м}$ удерживается за ее концы так, что они находятся на
	данной высоте рядом друг с другом. На нити висит проволочная скобка в виде перевернутой буквы
	$U$. Масса скобки равна $1$ грамму. Нить выдерживает максимальную растягивающую силу $F = 5 \unit{Н}$
	($F \gg mg$). Концы нити начинают перемещать в противоположных горизонтальных направлениях с
	одинаковыми скоростями $1 \unit{м/с}$. В какой-то момент нить не выдерживает и рвется. На какую
	максимальную высоту в момент разрыва нити взлетит скобка? Сопротивлением воздуха пренебречь.
}
% Туймаада, 1.17