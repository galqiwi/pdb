
\task{ Колонна машин движется по дороге, строго соблюдая определённый
  скоростной режим --- зависимость скорости машины от её положения $x$
  на дороге представлена на графике сплошной линией. Известно, что
  машины начинали своё движение с интервалом в $\tau=10$ с. В
  некоторый момент всем машинам одновременно поступило сообщение об
  ухудшении погодных условий, в соответствии с которым они должны
  изменить свой скоростной режим на другой --- изображённый
  прерывистой линией. Какой максимальный временной интервал будет
  наблюдаться между машинами, приходящими в конечный пункт? }

\begin{figure}[h]
  \centering
  \begin{tikzpicture}[/pgfplots/axis labels at tip/.style={ xlabel
      style={at={(current axis.right of origin)}, yshift=2 ex,
        anchor=east,fill=white}, ylabel style={at={(current axis.above
          origin)}, yshift=1.5ex, anchor=center}}]
    \begin{axis}[
      width=8cm,
      xmin=0,xmax=6,ymin=0,ymax=80,
      axis x line=bottom,
      axis y line=middle,
      axis labels at tip,
      xlabel={$x$, км},
      ylabel={$v$, км/ч},
      minor xtick={0.25,0.5,0.75,...,6},
      minor ytick={5,10,15,...,80},
      xtick={0,1,2,...,6},
      ytick={0,20,40,60},
      xticklabels={0,1,2,...,6},
      yticklabels={0,20,40,60},
      tick label style={font=\small},
      label style={font=\small},
      grid=both
      ]
      \addplot[very thick,red,mark=none] coordinates { (0,55)
        (0.75,55) (0.75,65) (1.75,65) (1.75,45) (3,45) (3,60) (4,60)
        (4,55) (5,55) (5,70) (6,70) };
      \addplot[very thick,dashed,blue,mark=none] coordinates { (0,40) (0.75,40)
        (0.75,45) (1.75,45) (1.75,35) (3,35) (3,40) (4,40) (4,50) (6,50)  };
    \end{axis}
  \end{tikzpicture}%
\end{figure}
% Город-2007, 8 класс
% ccpe-2016-2017-8